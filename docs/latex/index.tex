Elaborado por\+: Carlos David Mesa Martinez \href{mailto:cadmesa@unicauca.edu.co}{\texttt{ cadmesa@unicauca.\+edu.\+co}}

Los Sistemas de Control de Versiones (V\+CS) permiten guardar el rastro de las modificaciones sobre determinados elementos. En el contexto de este examen, se gestionarán versiones de archivos y directorios.

Se deberá implementar un sistema de control de versiones simple, que permita\+:

Adicionar un archivo al repositorio de versiones. Listar las versiones de un archivo en el repositorio de versiones. Obtener la versión de un archivo del repositorio de versiones.

En esta implementación sólo se deberá realizar el control de versiones por directorio, en el cual sólo se pueden agregar archivos que se encuentren en el directorio actual (no recursivo).

Uso\+: versions add A\+R\+C\+H\+I\+VO \char`\"{}\+Comentario\char`\"{} \+: Adiciona una version del archivo al repositorio versions list A\+R\+C\+H\+I\+VO \+: Lista las versiones del archivo existentes versions list \+: Lista todos los archivos en la base de datos (unsort) versions get numver A\+R\+C\+H\+I\+VO \+: Obtiene una version del archivo del repositorio versions version A\+R\+C\+H\+I\+VO \+: Obtiene la version del archivo que se esta trabajando

Ejemplo de uso\+: comandos\+: touch test.\+txt vim test.\+txt (agregar contenido al archivo test.\+txt) ls (verificar que el archivo se ha guardado en la carpeta actual) ./versions add test.\+txt \char`\"{}version numero uno de pruebas\char`\"{} ./versions list test.\+txt rm -\/r test.\+txt ./versions get 1 test.\+txt ls (ver si se recupero el archivo) cat test.\+txt (verificar el contenido recuperado de la version 1 del archivo test.\+txt) 