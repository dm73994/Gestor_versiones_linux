Elaborado por\+: Erwin Meza Vega \href{mailto:emezav@unicauca.edu.co}{\texttt{ emezav@unicauca.\+edu.\+co}}

Los Sistemas de Control de Versiones (V\+CS) permiten guardar el rastro de las modificaciones sobre determinados elementos. En el contexto de este examen, se gestionarán versiones de archivos y directorios.

Se deberá implementar un sistema de control de versiones simple, que permita\+:

Adicionar un archivo al repositorio de versiones. Listar las versiones de un archivo en el repositorio de versiones. Obtener la versión de un archivo del repositorio de versiones.

En esta implementación sólo se deberá realizar el control de versiones por directorio, en el cual sólo se pueden agregar archivos que se encuentren en el directorio actual.

Uso\+: versions add A\+R\+C\+H\+I\+VO \char`\"{}\+Comentario\char`\"{} \+: Adiciona una version del archivo al repositorio versions list A\+R\+C\+H\+I\+VO \+: Lista las versiones del archivo existentes versions get numver A\+R\+C\+H\+I\+VO \+: Obtiene una version del archivo del repositorio 